\documentclass{article}

\usepackage[utf8]{inputenc}

\usepackage{nicefrac}
\usepackage{amssymb, amsmath, amsfonts}
\usepackage{amsthm}
\usepackage{tikz}
\usetikzlibrary{matrix,shapes,arrows}
\usepackage{pgfplots}
\usepgfplotslibrary{groupplots}
\usepackage[a4paper, margin=1in]{geometry}

\newtheorem{proposition}{Proposition}
\newtheorem{theorem}{Theorem}
\newtheorem{definition}{Definition}
\newtheorem{lemma}{Lemma}
\newtheorem{conjecture}{Conjecture}
\newtheorem{corollary}{Corollary}
\newtheorem{remark}{Remark}
\newtheorem{assumption}{Assumption}

\newlength\figureheight
\newlength\figurewidth
\setlength\figureheight{8cm}
\setlength\figurewidth{14cm}

\newcommand{\tikzdir}[1]{tikz/#1.tikz}
\newcommand{\inputtikz}[1]{\input{\tikzdir{#1}}}

\DeclareMathOperator*{\argmin}{arg\; min}     % argmin
\DeclareMathOperator*{\argmax}{arg\; max}     % argmax
\DeclareMathOperator*{\tr}{tr}     % trace
\DeclareMathOperator{\Cov}{Cov}
\DeclareMathOperator{\logdet}{log\;det}

\title{EE3011 Modeling and Control\\Quiz 4}
\date{}
\begin{document} \maketitle

\begin{enumerate}
\item The Bode magnitude plot of a minimum phase system is shown in Figure~\ref{fig:1}. Determine an approximate mathematical model in the form of transfer function for the system.
  \begin{figure}[ht]
    \setlength\figureheight{8cm}
    \setlength\figurewidth{14cm}
    \centering
    \inputtikz{Quiz21}
    \caption{Magnitude Plot\label{fig:1}}
  \end{figure}

  {\bf Solution:}
  \begin{enumerate}
  \item The initial slope is $-20$: $1/s$
  \item The low frequency line passes through $(0.5,20dB)$:
    \begin{align*}
      20\log K - 20 \log 0.5 = 20\Rightarrow K = 5.
    \end{align*}
  \item The slope change to 0 at frequency $0.5$: $(s/0.5+1)$
  \item The slope change to -20 at frequency $5$: $(s/5+1)^{-1}$
  \item The slope change to -60 at frequency $50$, the correction is $0dB$:
    \begin{align*}
      \omega_n = 50,\,-20\log 2\zeta = 0 \Rightarrow \zeta = 0.5.
    \end{align*}
  \end{enumerate}
\begin{align*}
  G(s) = \frac{5(s/0.5+1)}{s(s/5+1)\left[(s/50)^2+s/50+1\right]}.
\end{align*}

  \newpage
\item Consider a feedback system with the following open-loop transfer function:
  \begin{align*}
    G(s) = K\frac{s+1}{s^2+4s-5}.
  \end{align*}
  The Bode plots of $G(s)$ when $K = 1$ are shown in Figure~\ref{fig:2}. Sketch the Nyquist plot. Using Nyquist Stability Criterion, find the range of $K$ that makes the system closed-loop stable.
  \begin{figure}[ht]
    \setlength\figureheight{4cm}
    \setlength\figurewidth{14cm}
    \centering
    \inputtikz{Quiz22}
    \caption{Bode Plot of $(s+1)/(s^2+4s-5)$\label{fig:2}}
  \end{figure}

  {\bf Solution:}
  \begin{enumerate}
  \item $\omega = 0$, $G(j\omega) = 0.2\angle 180^\circ$.
  \item ({\it optional}): $\angle G(j\omega) = -90$ implies that the real part of $G(j\omega) = 0$. Hence,
    \begin{align*}
      \mathfrak{Re}\left[(j\omega + 1)(-5-\omega^2 -4j\omega)\right] = 0 \Rightarrow \omega = \sqrt{5/3}.
    \end{align*}
    The corresponding point is $G(j\omega) = 0.1937\angle -90^\circ$.
  \item $\omega\rightarrow \infty$, $G(j\omega) = 0$
  \item Segment 2: origin; Segment 3: mirror reflection.

  \item Poles for the open loop transfer function: -5 (stable), 1 (unstable). $P=1$
  \item  If $K > 5$, the Nyquist plot encircles $-1$ counterclockwise once. Hence $N = -1$, $Z = N+P = 0$, the system is closed-loop stable.
  \item  If $K < 5$, the Nyquist plot does not encircle $-1$. Hence $N = 0$, $Z = N+P = 1$, the system is closed-loop unstable.
  \end{enumerate}

  \begin{figure}[h]
    \setlength\figureheight{5cm}
    \setlength\figurewidth{8cm}
    \centering
    \inputtikz{Quiz22Nyquist}
  \end{figure}


  \newpage
\item The Bode magnitude plot of a minimum phase system is shown in Figure~\ref{fig:1}. Determine an approximate mathematical model in the form of transfer function for the system.
  \begin{figure}[ht]
    \setlength\figureheight{8cm}
    \setlength\figurewidth{14cm}
    \centering
    \inputtikz{Quiz23}
    \caption{Magnitude Plot\label{fig:1}}
  \end{figure}

  {\bf Solution:}
  \begin{enumerate}
  \item The initial slope is $-20$: $1/s$
  \item The low frequency line passes through $(0.2,20dB)$:
    \begin{align*}
      20\log K - 20 \log 0.2 = 20\Rightarrow K = 2.
    \end{align*}
  \item The slope change to -40 at frequency $0.2$: $(s/0.2+1)^{-1}$
  \item The slope change to 0 at frequency $2$, the correction is $-14dB$:
    \begin{align*}
      \omega_n = 2,\,20\log 2\zeta = -14 \Rightarrow \zeta = 0.1.
    \end{align*}
  \item The slope change to -20 at frequency $5$: $(s/20+1)^{-1}$
  \end{enumerate}
\begin{align*}
  G(s) = \frac{2\left[(s/2)^2+0.2\times s/2+1\right]}{s(s/20+1)(s/0.2+1)}.
\end{align*}

  \newpage
\item Consider a feedback system with the following open-loop transfer function:
  \begin{align*}
    G(s) = K\frac{s-1}{s^2+2s+1}.
  \end{align*}
  The Bode plots of $G(s)$ when $K = 1$ are shown in Figure~\ref{fig:2}. Sketch the Nyquist plot. Using Nyquist Stability Criterion, find the range of $K$ that makes the system closed-loop stable.
  \begin{figure}[ht]
    \setlength\figureheight{4cm}
    \setlength\figurewidth{14cm}
    \centering
    \inputtikz{Quiz24}
    \caption{Bode Plot of $(s-1)/(s^2+2s+1)$\label{fig:2}}
  \end{figure}
  {\bf Solution:}
  \begin{enumerate}

  \item $\omega = 0$, $G(j\omega) = 1\angle 180^\circ$.
  \item ({\it optional}): $\angle G(j\omega) = 90$ implies that the real part of $G(j\omega) = 0$. Hence,
    \begin{align*}
      \mathfrak{Re}\left[(j\omega - 1)(1-\omega^2 -2j\omega)\right] = 0 \Rightarrow \omega = \sqrt{1/3}.
    \end{align*}
    The corresponding point is $G(j\omega) = 0.8660\angle 90^\circ$.
  \item ({\it optional}): $\angle G(j\omega) = 0$ implies that the imaginary part of $G(j\omega) = 0$. Hence,
    \begin{align*}
      \mathfrak{Im}\left[(j\omega - 1)(1-\omega^2 -2j\omega)\right] = 0 \Rightarrow \omega = \sqrt{3}.
    \end{align*}
    The corresponding point is $G(j\omega) = 0.5\angle 0^\circ$.
  \item $\omega\rightarrow \infty$, $G(j\omega) = 0$
  \item Segment 2: origin; Segment 3: mirror reflection.

  \item Poles for the open loop transfer function: -1 (stable), -1 (stable). $P=0$
  \item  If $K > 1$, the Nyquist plot encircles $-1$ clockwise once. Hence $N = 1$, $Z = N+P = 1$, the system is closed-loop unstable.
  \item  If $K < 1$, the Nyquist plot does not encircle $-1$. Hence $N = 0$, $Z = N+P = 0$, the system is closed-loop stable.
  \end{enumerate}

  \begin{figure}[h]
    \setlength\figureheight{4cm}
    \setlength\figurewidth{8cm}
    \centering
    \inputtikz{Quiz24Nyquist}
  \end{figure}
  \newpage
\item The Bode magnitude plot of a minimum phase system is shown in Figure~\ref{fig:1}. Determine an approximate mathematical model in the form of transfer function for the system.
  \begin{figure}[ht]
    \setlength\figureheight{8cm}
    \setlength\figurewidth{14cm}
    \centering
    \inputtikz{Quiz25}
    \caption{Magnitude Plot\label{fig:1}}
  \end{figure}

  {\bf Solution:}
  \begin{enumerate}
  \item The initial slope is $-20$: $1/s$
  \item The low frequency line passes through $(0.5,20dB)$:
    \begin{align*}
      20\log K - 20 \log 0.5 = 20\Rightarrow K = 5.
    \end{align*}
  \item The slope change to 0 at frequency $0.5$: $(s/0.5+1)$
  \item The slope change to 20 at frequency $5$: $(s/5+1)$
  \item The slope change to -20 at frequency $20$, the correction is $8dB$:
    \begin{align*}
      \omega_n = 20,\,-20\log 2\zeta = 8 \Rightarrow \zeta = 0.2.
    \end{align*}
  \end{enumerate}
\begin{align*}
  G(s) = \frac{5(s/5+1)(s/0.5+1)}{s\left[(s/20)^2+0.4\times s/20+1\right]}.
\end{align*}


  \newpage
\item Consider a feedback system with the following open-loop transfer function:
  \begin{align*}
    G(s) = K\frac{s-3}{s^2-4s+4}.
  \end{align*}
  The Bode plots of $G(s)$ when $K = 1$ are shown in Figure~\ref{fig:2}. Sketch the Nyquist plot. Using Nyquist Stability Criterion, determine whether there exists a $K > 0$ that makes the system closed-loop stable. If so, find the range of the stabilizing $K$.
  \begin{figure}[ht]
    \setlength\figureheight{5cm}
    \setlength\figurewidth{14cm}
    \centering
    \inputtikz{Quiz26}
    \caption{Bode Plot of $(s-3)/(s^2-4s+4)$\label{fig:2}}
  \end{figure}

  {\bf Solution:}
  \begin{enumerate}

  \item $\omega = 0$, $G(j\omega) = 0.75\angle 180^\circ$. Since the phase is increasing, the plot goes counterclockwise.
  \item No real or imaginary intersections.
  \item $\omega\rightarrow \infty$, $G(j\omega) = 0$
  \item Segment 2: origin; Segment 3: mirror reflection.
  \item Poles for the open loop transfer function: 2 (unstable), 2 (unstable). $P=2$
  \item  If $K > 4/3$, the Nyquist plot encircles $-1$ counterclockwise once. Hence $N = -1$, $Z = N+P = 1$, the system is closed-loop unstable.
  \item  If $K < 4/3$, the Nyquist plot does not encircle $-1$. Hence $N = 0$, $Z = N+P = 2$, the system is closed-loop unstable.
  \end{enumerate}

  \begin{figure}[h]
    \setlength\figureheight{5.cm}
    \setlength\figurewidth{8cm}
    \centering
    \inputtikz{Quiz26Nyquist}
  \end{figure}

\end{enumerate}

\end{document}
%%% Local Variables:
%%% TeX-command-default: "Latexmk"
%%% End:
