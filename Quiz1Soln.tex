\documentclass{article}

\usepackage[utf8]{inputenc}

\usepackage{nicefrac}
\usepackage{amssymb, amsmath, amsfonts}
\usepackage{amsthm}
\usepackage{tikz}
\usetikzlibrary{matrix,shapes,arrows}
\usepackage{pgfplots}
\usepgfplotslibrary{groupplots}
\usepackage[a4paper, margin=1in]{geometry}

\newtheorem{proposition}{Proposition}
\newtheorem{theorem}{Theorem}
\newtheorem{definition}{Definition}
\newtheorem{lemma}{Lemma}
\newtheorem{conjecture}{Conjecture}
\newtheorem{corollary}{Corollary}
\newtheorem{remark}{Remark}
\newtheorem{assumption}{Assumption}

\newlength\figureheight
\newlength\figurewidth
\setlength\figureheight{6cm}
\setlength\figurewidth{14cm}

\newcommand{\tikzdir}[1]{tikz/#1.tikz}
\newcommand{\inputtikz}[1]{\input{\tikzdir{#1}}}

\DeclareMathOperator*{\argmin}{arg\; min}     % argmin
\DeclareMathOperator*{\argmax}{arg\; max}     % argmax
\DeclareMathOperator*{\tr}{tr}     % trace
\DeclareMathOperator{\Cov}{Cov}
\DeclareMathOperator{\logdet}{log\;det}

\title{EE3011 Modeling and Control\\Quiz 3}
\date{}
\begin{document} \maketitle

\begin{enumerate}
\item Find the magnitude and phase responses of the following transfer function
  \[
    G(s) = \frac{s-1}{s+1}.
  \]
  Compute the steady-state responses to sinusoid $r(t) = 2\sin(t+30^\circ)$.

  {\bf Solution:} Notice that
  \begin{align*}
    -1+j\omega = \sqrt{1+\omega^2}\angle (180^\circ-\tan^{-1}\omega),\,1+j\omega = \sqrt{1+\omega^2}\angle \tan^{-1}\omega.
  \end{align*}
  Therefore,
  \begin{align*}
    G(j\omega) = 1\angle (180^\circ-2\tan^{-1}\omega).
  \end{align*}
  For the sinusoid signal, $\omega = 1$. Thus,
  \begin{align*}
    G(1j) = 1\angle (180^\circ-2\times 45^\circ) = 1\angle 90^\circ.
  \end{align*}
  The output is $y(t) = 2\sin(t+120^\circ)$.


\newpage

\item Sketch the Bode plots for the following transfer function
  \[
    G(s) = \frac{10(s+50)}{s(s+5)}.
  \]
  {\bf Solution:} \begin{enumerate}
  \item The normalized transfer function is
  \[
    G(s) = \frac{100(s/50+1)}{s(s/5+1)}.
  \]
    
\item The low frequency part is $100/s$. The magnitude plot is a line with slope -20 and passing through (1,40dB). The phase plot is a horizontal line at $-90^\circ$.
\item The magnitude plot:
  \begin{table}[h]
    \centering
    \begin{tabular}{c|ccc}
      & low & 5   & 50  \\
      \hline
      slope change &     & -20 & +20 \\
      slope        & -20 & -40 & -20
    \end{tabular}
  \end{table}
\item The phase plot:
  \begin{table}[h]
    \centering
    \begin{tabular}{c|ccccc}
      & low & 0.5 & 5   & 50  & 500 \\
      \hline
      slope change &     & -45 & +45 & +45 & -45 \\
      slope        & 0   & -45 & 0   & +45 & 0  
    \end{tabular}
  \end{table}
\end{enumerate}

\begin{figure}[h]
  \centering
  \inputtikz{Quiz12}
\end{figure}
\newpage  
\item Find the magnitude and phase responses of the following transfer function
  \[
    G(s) = \frac{1}{(0.1s+1)^3}.
  \]
  Compute the steady-state responses to sinusoid $r(t) = 4\sin(10\sqrt{3}t+60^\circ)$.

  {\bf Solution:} Notice that 
  \begin{align*}
    1+0.1\omega j = \sqrt{1+0.01\omega^2} \angle \tan^{-1}(0.1\omega).
  \end{align*}
  Therefore,
  \begin{align*}
    G(j\omega) = \left(1+0.01\omega^2\right)^{-3/2}\angle -3\tan^{-1}(0.1\omega).
  \end{align*}
  For the sinusoid signal, $\omega = 10\sqrt{3}$. Hence,
  \begin{align*}
    G(10\sqrt{3}j) = \left(1+3\right)^{-3/2}\angle (-3\times 60^\circ) = 0.125\angle -180^\circ.
  \end{align*}
  The output is $y(t) = 0.5 \sin(10\sqrt{3}t -120^\circ)$.
  \newpage

\item Sketch the Bode plots for the following transfer function
  \[
    G(s) = \frac{5(s+20)}{s^2}.
  \]
 \begin{enumerate}
  \item The normalized transfer function is
  \[
    G(s) = \frac{100(s/20+1)}{s^2}.
  \]
    
\item The low frequency part is $100/s^2$. The magnitude plot is a line with slope -40 and passing through (1,40dB). The phase plot is a horizontal line at $-180^\circ$.
\item The magnitude plot:
  \begin{table}[h]
    \centering
    \begin{tabular}{c|cc}
      & low &  20  \\
      \hline
      slope change &     &  +20 \\
      slope        & -40 &  -20
    \end{tabular}
  \end{table}
\item The phase plot:
  \begin{table}[h]
    \centering
    \begin{tabular}{c|ccccc}
      & low &   2   &  200 \\
      \hline
      slope change &     &  +45 & -45 \\
      slope        & 0   &  +45 & 0  
    \end{tabular}
  \end{table}
\end{enumerate}
\begin{figure}[h]
  \centering
  \inputtikz{Quiz14}
\end{figure}
\end{enumerate}

\end{document}
%%% Local Variables:
%%% TeX-command-default: "Latexmk"
%%% End:
