\documentclass{article}

\usepackage[utf8]{inputenc}

\usepackage{nicefrac}
\usepackage{amssymb, amsmath, amsfonts}
\usepackage{amsthm}
\usepackage{tikz}
\usetikzlibrary{matrix,shapes,arrows}
\usepackage{pgfplots}
\usepgfplotslibrary{groupplots}
\usepackage[a4paper, margin=1in]{geometry}

\newtheorem{proposition}{Proposition}
\newtheorem{theorem}{Theorem}
\newtheorem{definition}{Definition}
\newtheorem{lemma}{Lemma}
\newtheorem{conjecture}{Conjecture}
\newtheorem{corollary}{Corollary}
\newtheorem{remark}{Remark}
\newtheorem{assumption}{Assumption}

\newlength\figureheight
\newlength\figurewidth
\setlength\figureheight{5.5cm}
\setlength\figurewidth{14cm}

\newcommand{\tikzdir}[1]{tikz/#1.tikz}
\newcommand{\inputtikz}[1]{\input{\tikzdir{#1}}}

\DeclareMathOperator*{\argmin}{arg\; min}     % argmin
\DeclareMathOperator*{\argmax}{arg\; max}     % argmax
\DeclareMathOperator*{\tr}{tr}     % trace
\DeclareMathOperator{\Cov}{Cov}
\DeclareMathOperator{\logdet}{log\;det}

\title{EE3011 Modeling and Control\\Tutorial 10: Nyquist Stability Criterion}
\date{}
\begin{document} \maketitle
Consider feedback systems with the following open-loop transfer functions:
\begin{enumerate}
\item \begin{align*}
        \frac{-K(s^2+0.5s+1)}{s^2+s+1}.
      \end{align*}

\item \begin{align*}
        \frac{K(s-1)}{(s-2)(s-4)}.
      \end{align*}
\item \begin{align*}
        \frac{K}{(s+1)(3s+1)(0.4s+1)}.
      \end{align*}
\item \begin{align*}
        \frac{K(s+3)}{s^2-4}.
      \end{align*}
\end{enumerate}


The Bode plots of each transfer function when $K = 1$ are shown below. Sketch the Nyquist plots and determine the range of $K$ for which the closed-loop system is stable.


  \begin{figure}[h]
    \centering
    \inputtikz{Tut101}
    \caption{abc}
  \end{figure}

  \begin{figure}[h]
    \centering
    \inputtikz{Tut102}
    \caption{abc}
  \end{figure}
  \begin{figure}[h]
    \centering
    \inputtikz{Tut103}
    \caption{abc}
  \end{figure}
  \begin{figure}[h]
    \centering
    \inputtikz{Tut104}
    \caption{abc}
  \end{figure}

\end{document}
%%% Local Variables:
%%% TeX-command-default: "Latexmk"
%%% End:
