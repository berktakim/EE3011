\documentclass{article}

\usepackage[utf8]{inputenc}

\usepackage{nicefrac}
\usepackage{amssymb, amsmath, amsfonts}
\usepackage{amsthm}
\usepackage{tikz}
\usetikzlibrary{matrix,shapes,arrows}
\usepackage{pgfplots}
\usepgfplotslibrary{groupplots}
\usepackage[a4paper, margin=1in]{geometry}

\newtheorem{proposition}{Proposition}
\newtheorem{theorem}{Theorem}
\newtheorem{definition}{Definition}
\newtheorem{lemma}{Lemma}
\newtheorem{conjecture}{Conjecture}
\newtheorem{corollary}{Corollary}
\newtheorem{remark}{Remark}
\newtheorem{assumption}{Assumption}

\newlength\figureheight
\newlength\figurewidth
\setlength\figureheight{7cm}
\setlength\figurewidth{14cm}

\newcommand{\tikzdir}[1]{tikz/#1.tikz}
\newcommand{\inputtikz}[1]{\input{\tikzdir{#1}}}

\DeclareMathOperator*{\argmin}{arg\; min}     % argmin
\DeclareMathOperator*{\argmax}{arg\; max}     % argmax
\DeclareMathOperator*{\tr}{tr}     % trace
\DeclareMathOperator{\Cov}{Cov}
\DeclareMathOperator{\logdet}{log\;det}

\title{EE3011 Modeling and Control\\Tutorial 8: Bode Plots/Frequency Domain Modeling}
\date{}
\begin{document} \maketitle

\begin{enumerate}
\item Sketch the Bode plots of the following simple transfer functions:
  \begin{enumerate}
  \item  \[G_1(s)=\frac{100(s+1)}{s+10}.\]
  \item  \[G_2(s)=\frac{100}{(s+2)(s+20)}.\]
  \item  \[G_3(s) = \frac{100}{s(s^2+2s+100)}.\]
  \item  \[G_4(s) = \frac{s^2+2\zeta_1\omega_ns+\omega_n^2}{s^2+2\zeta_2\omega_ns+\omega_n^2},\] where $\zeta_1 = 0.1,\,\zeta_2 = 0.6,\,\omega_n = 10$. 
  \end{enumerate}
\item The magnitude response plot and its asymptotes of a minimum phase system are shown in Figure~\ref{fig:2}. Which of the following transfer functions best approximates the frequency response of the system? Explain your answer.
\begin{enumerate}
\item $\nicefrac{50000s}{(s^2+60s+500)}$;
\item $\nicefrac{100(s+10)}{(s^2+50s)}$;
\item $\nicefrac{500(s+10)}{(s^2+50s)}$.
\end{enumerate}

  \begin{figure}[ht]
    \centering
    \inputtikz{Tut82}
    \caption{Magnitude Plot\label{fig:2}}
  \end{figure}
  \newpage
\item The magnitude response of a first order system is given in Figure~\ref{fig:3}. Find the gain and the time constant of the system.

  \begin{figure}[ht]
    \centering
    \inputtikz{Tut83}
    \caption{Magnitude Plot\label{fig:3}}
  \end{figure}
\end{enumerate}
\newpage

\section*{Answers:}
\begin{enumerate}
\item 
  \begin{enumerate}
  \item  \[G_1(s)=\frac{10(s+1)}{0.1s+1}.\]
  \begin{figure}[ht]
    \centering
    \inputtikz{Tut81a}
  \end{figure}
  \newpage
  \item  \[G_2(s)=\frac{2.5}{(0.5s+1)(0.05s+1)}.\]
  \begin{figure}[ht]
    \centering
    \inputtikz{Tut81b}
  \end{figure}
  \newpage
  \item  \[G_3(s) = \frac{1}{s((s/10)^2+2\times 0.1 \times (s/10)+1)}.\]
  \begin{figure}[ht]
    \centering
    \inputtikz{Tut81c}
  \end{figure}
  \newpage
  \item  \[G_4(s) = \frac{(s/10)^2+2\times 0.1(s/10)+1}{(s/10)^2+2\times 0.6(s/10)+1}.\] 
  \begin{figure}[ht]
    \centering
    \inputtikz{Tut81d}
  \end{figure}
  \end{enumerate}

\item (c)
\item $G(s) = 10/(0.1s+1)$. Gain is 10 and time constant is 10.
\end{enumerate}
\end{document}
%%% Local Variables:
%%% TeX-command-default: "Latexmk"
%%% End: